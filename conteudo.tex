\section{Introdução}

A introdução é uma das seções mais importantes do artigo, pois estabelece o contexto do problema e motiva a pesquisa. Ela deve conter:
\begin{itemize}
    \item Contexto do problema: Explique a área de pesquisa e o problema específico abordado. Relacione com situações do mundo real quando apropriado.

\item  Relevância: Justifique por que o problema merece ser estudado. Apresente estatísticas ou exemplos que demonstrem sua importância prática ou teórica.

\item Objetivos do estudo: Descreva o que o artigo busca resolver, responder ou contribuir.

\item Estrutura do artigo: Dê uma visão geral das próximas seções, explicando brevemente o conteúdo de cada uma.
\end{itemize}
\subsection{Uso de Overleaf e do Latex}

O uso do \textbf{Overleaf} e do \textbf{LaTeX} é altamente benéfico para alunos de mestrado, especialmente aqueles que precisam produzir dissertações, artigos científicos e relatórios técnicos de alta qualidade. Diferente dos processadores de texto convencionais, o \LaTeX\ permite um controle preciso da formatação, garantindo a correta estruturação de referências, equações matemáticas e tabelas complexas. Além disso, ele segue padrões acadêmicos reconhecidos internacionalmente, facilitando a submissão de trabalhos para conferências e periódicos.  

O Overleaf, por sua vez, é uma plataforma colaborativa baseada em nuvem que simplifica o uso do \LaTeX\, eliminando a necessidade de instalar pacotes localmente. Isso possibilita que alunos e orientadores trabalhem simultaneamente no mesmo documento, facilitando revisões e ajustes em tempo real. Além disso, o Overleaf conta com templates prontos para diversas universidades e publicações científicas, o que agiliza a escrita acadêmica ao reduzir o tempo gasto com ajustes de formatação.  

Outro grande benefício do \LaTeX\ é a sua capacidade de lidar com grandes quantidades de referências bibliográficas por meio do BibTeX. Isso evita problemas comuns em editores de texto tradicionais, como erros na numeração ou inconsistências no estilo de citação. Para alunos de mestrado que precisam integrar diversas fontes de pesquisa e garantir a padronização da bibliografia, essa funcionalidade é essencial.  

Além da qualidade tipográfica superior, o \LaTeX\ também é muito útil para a escrita de documentos técnicos com grande quantidade de fórmulas matemáticas, gráficos e algoritmos. Diferente de editores visuais, que podem apresentar problemas na exibição desses elementos, o \LaTeX\ garante que tudo seja representado corretamente, independentemente do dispositivo ou plataforma onde o documento for aberto.  

Por fim, ao utilizar o Overleaf e \LaTeX\, o aluno de mestrado desenvolve habilidades valiosas para a pesquisa acadêmica e a produção científica, preparando-se melhor para um futuro na academia ou na indústria. O domínio dessas ferramentas proporciona mais eficiência, organização e profissionalismo na apresentação dos trabalhos, tornando-os mais claros e acessíveis para a comunidade científica.

\section{Trabalhos Relacionados}
Esta seção é essencial para situar seu trabalho na literatura existente. Deve incluir:
\begin{itemize}
    \item Revisão de literatura: Apresente estudos anteriores relevantes sobre o tema, citando referências acadêmicas recentes.

    \item Comparativo com seu trabalho: Explique como seu estudo se diferencia dos existentes. Destaque as limitações das soluções anteriores e como sua abordagem propõe melhorias.

    \item Organização lógica: Agrupe os trabalhos relacionados em categorias (exemplo: por abordagem, por tipo de problema resolvido, etc.) para facilitar a leitura e a compreensão.

\end{itemize}

\section{Metodologia}

Nesta seção, explique como sua pesquisa foi conduzida. Isso garante reprodutibilidade e transparência. Inclua:

\begin{itemize}
    \item Abordagem adotada: Explique se o estudo é experimental, teórico, simulação ou baseado em análise de dados.

    \item Detalhamento dos métodos: Descreva os algoritmos, modelos matemáticos ou abordagens utilizadas.

    \item Ferramentas e Tecnologias: Cite linguagens de programação, frameworks e hardware utilizados.

    \item Bases de dados: Se aplicável, informe quais bases de dados foram usadas para validação.

    \item Critérios de avaliação: Defina métricas utilizadas para avaliar os resultados (exemplo: tempo de execução, precisão, eficiência).
\end{itemize}

\subsection{Citações}

    Utilizaremos o formato de citação autor-ano, assegurando a padronização das referências bibliográficas ao longo do documento e facilitando a identificação das fontes citadas no texto.

    Nesta frase, apresentamos um exemplo de citação indireta:

    Segundo \cite{bianchi2021miint}, o uso da técnica X é recomendado por apresentar vantagens significativas, como a melhoria da eficiência computacional e a redução da complexidade na implementação de determinados processos. Essa abordagem tem sido amplamente adotada em diferentes contextos devido à sua eficácia na resolução de problemas específicos. Posteriormente, uma abordagem alternativa foi descrita por pesquisadores em \cite{vadlamudi2023implementing}, os quais propuseram uma metodologia distinta que aprimora determinados aspectos do desempenho e da escalabilidade, possibilitando sua aplicação em cenários mais complexos e exigentes.

    Quando a citação for maior que três linhas, devemos formatá-la como uma citação em bloco, conforme o exemplo a seguir:  

    \begin{citacao}
    Lorem ipsum dolor sit amet, consectetuer adipiscing elit. Ut purus elit, vestibulumut, placerat ac, adipiscing vitae, felis. Curabitur dictum gravida mauris. Nam arcu libero,nonummy eget, consectetuer id, vulputate a, magna. Donec vehicula augue eu neque. \cite{lamnaour2024semantic}
    \end{citacao}

    As referências estão no arquivo \texttt{referencias.bib}, no formato BibTeX. Vários portais de artigos disponibilizam a citação nesse formato, incluindo o próprio Google Scholar. O BibTeX facilita a organização e a padronização das referências bibliográficas, permitindo sua inserção automática no documento.  

    A seguir, estão listados alguns links explicando esse formato e como utilizá-lo corretamente no Overleaf:
    \begin{itemize}
        \item 
        \textbf{Using BibTeX}: Este guia fornece instruções detalhadas sobre como utilizar o BibTeX, incluindo a criação de arquivos \texttt{.bib} e a inserção de referências em documentos LaTeX~\footnote{\url{https://www.bibtex.org/Using/}}; 
        \item 
        \textbf{BibTeX Format Explained [with examples]}: Este artigo explica a estrutura do formato BibTeX, detalhando os tipos de entradas e campos disponíveis, acompanhado de exemplos práticos\footnote{\url{https://www.bibtex.com/g/bibtex-format}}; 
        \item 
        Bibliography Management with BibTeX: O Overleaf oferece um tutorial abrangente sobre como gerenciar bibliografias utilizando BibTeX, abordando desde a criação de arquivos de bibliografia até a configuração de estilos de citação\footnote{\url{https://www.overleaf.com/learn/latex/Bibliography_management_with_bibtex}}.
    \end{itemize}

\subsection{Figuras}

Para incluir figuras, como a Figura~\ref{fig:exemplo-figura}, podemos utilizar formatos comuns de imagens, como JPG ou PNG. Entretanto, dê preferência a figuras vetorizadas, no formato PDF, assim a qualidade não será comprometida e o arquivo PDF final terá um tamanho menor. 

\begin{figure}[!htbp]
    \centering
    \caption{\label{fig:exemplo-figura} Arquitetura de Software do \textit{Middleware} EXEHDA.}
    \includegraphics[width=\textwidth]{img/ArquiteturaExehda_v4-pt_BR.png}
    
    \legend{Fonte: Elaborada pelo autor, adaptada de \cite{lopes2014middleware}}
\end{figure}

As figuras no \LaTeX\ nem sempre são posicionadas exatamente onde foram inseridas no código. Em vez disso, o algoritmo interno ajusta automaticamente a disposição dos elementos para garantir a melhor apresentação do documento. Portanto, evite expressões como ``de acordo com a figura abaixo'', pois a figura pode não estar imediatamente abaixo. Em vez disso, utilize referências automáticas, como: ``de acordo com a \autoref{fig:exemplo-figura}''.

Para evitar que um parágrafo seja interrompido por figuras, você pode utilizar o comando \verb|\FloatBarrier|. Caso esteja lidando com várias figuras seguidas e deseje forçar sua exibição antes de prosseguir para a próxima seção do texto, o comando \verb|\clearpage| pode ser útil, garantindo que todas as figuras pendentes sejam renderizadas antes da nova página.

\subsubsection{Figuras em minipages}
\emph{Minipages} são utilizadas para organizar textos, imagens ou outros elementos dentro de quadros com tamanhos e posições controladas. Esse recurso é especialmente útil para alinhar conteúdos lado a lado sem depender da estrutura padrão de colunas. Veja, por exemplo, a \autoref{fig_minipage_imagem1} e a \autoref{fig_minipage_grafico2}.

\begin{figure}[htb]
 \label{teste}
 \centering
  \begin{minipage}{0.4\textwidth}
    \centering
    \caption{ESP 8266} \label{fig_minipage_imagem1}
    \includegraphics[scale=0.4]{img/esp32.jpeg}
    \legend{Fonte: Produzido pelos autores}
  \end{minipage}
  \hfill
  \begin{minipage}{0.4\textwidth}
    \centering
    \caption{ESP 32} \label{fig_minipage_grafico2}
    \includegraphics[scale=0.38]{img/esp8266.jpeg}
    \vspace{2mm}
    \legend{Fonte: \cite{nunes2011associations} }
  \end{minipage}
\end{figure}

\subsubsection{Sub-figuras}
Sub-figuras em \LaTeX\ são figuras organizadas dentro de uma figura principal, permitindo que várias imagens sejam apresentadas lado a lado ou empilhadas dentro de um mesmo ambiente, vide figura~\ref{fig:exemplo-subfiguras}. Isso é útil para comparar gráficos, diagramas ou imagens relacionadas sem precisar criar múltiplos ambientes figure.

\begin{figure}[!htbp]
    \centering
    \caption{\label{fig:exemplo-subfiguras} Exemplo de subfiguras.}
    \begin{subfigure}[b]{0.3\textwidth}  % tamanho da subfigura
        \centering
        \includegraphics[width=\textwidth]{img/logo_UCPEL.png}
        \caption{\label{fig:subfig1} subfigura à esquerda}
    \end{subfigure}
    \hfill
    \begin{subfigure}[b]{0.3\textwidth}  % tamanho da subfigura
        \centering
        \includegraphics[width=\textwidth]{img/logo_UCPEL_PRETO.png}
        \caption{\label{fig:subfig2} subfigura à direita}
    \end{subfigure}
\end{figure}

\subsection{Tabelas}

Para a criação de tabelas em \LaTeX\, como a~\autoref{tab:gpio} e a~\autoref{tab:algoritmo-compara}, é possível utilizar a ferramenta online \url{https://www.tablesgenerator.com}, que permite elaborar tabelas de forma visual e intuitiva. Essa ferramenta facilita a formatação dos dados, possibilitando a personalização de estilos, alinhamentos e bordas, além de gerar automaticamente o código correspondente em \LaTeX\. Dessa forma, a utilização dessa abordagem simplifica a construção de tabelas complexas, garantindo maior precisão na apresentação dos dados e otimização do tempo na edição do documento.

\begin{table}[!ht]
\centering
\footnotesize
\caption{GPIOS}
%\vspace{5mm}
\label{tab:gpio}
\begin{tabular}{l|l}
\hline
ESP 32      & SIM 800L \\ \hline
GPIO 2 (TX) & RX       \\ \hline
GPIO 4 (RX) & TX       \\ \hline
GND         & GND      \\ \hline
\end{tabular}

\vspace*{3mm}
{\small Fonte: Preparada pelo autor.}     
\end{table}

As tabelas em \LaTeX\ permitem a organização clara e estruturada de dados, oferecendo ampla flexibilidade na formatação, alinhamento e personalização, garantindo uma apresentação precisa e profissional em documentos técnicos e acadêmicos, como ilustrado na \autoref{tab:algoritmo-compara}, onde a disposição dos dados segue um padrão bem definido para facilitar a leitura e interpretação das informações.


\begin{table}[h]
\small
\centering
\captionsetup{justification=centering}
\caption{Comparison of Association Rule Algorithms}
\label{tab:algoritmo-compara}
\begin{tabular}{|l|l|l|l|}
\hline
\textbf{Algorithm} & \begin{tabular}[c]{@{}l@{}}\textbf{Ease} \\ \textbf{Deployment}  \end{tabular} & \begin{tabular}[c]{@{}l@{}}\textbf{Big Data} \\\textbf{Efficiency} \end{tabular} & \begin{tabular}[c]{@{}l@{}}\textbf{Computational } \\ \textbf{Complexity}\end{tabular} \\ \hline
Apriori & High & Medium & High \\ \hline
FP-Growth & Medium & High & Medium \\ \hline
Eclat & Medium & \begin{tabular}[c]{@{}l@{}}Alta\end{tabular} & \begin{tabular}[c]{@{}l@{}}Variável \end{tabular} \\ \hline
RARM & Low & High & High \\ \hline
STUCCO & Low & Medium & High \\ \hline
\end{tabular}

\vspace*{3mm}
{\small Source: Prepared by the author.}        


\end{table}

\subsection{Equações}

Um exemplo de equação destacada pode ser observado na \autoref{eq:exemplo-1}, enquanto um exemplo de equação inserida diretamente no meio do parágrafo é dado por: $v_j(n+1) = v_j(n) + \Delta v_j(n)$ . Caso ainda não esteja familiarizado com a sintaxe matemática do \LaTeX\ ou não conheça o nome exato de um determinado símbolo matemático, é recomendável utilizar a ferramenta online \url{https://editor.codecogs.com} Esse editor permite a construção de equações de forma visual e intuitiva, semelhante ao Editor de Equações do Microsoft Word, facilitando a criação e formatação de expressões matemáticas complexas sem a necessidade de memorizar comandos específicos.

\begin{equation}\label{eq:exemplo-1}
    v_j(n+1) = v_j(n) + \Delta v_j(n) =  v_j(n) - \eta_v\cdot\nabla_{v_j}E(n) = v_j(n) - \eta_v\cdot\frac{\partial E(n)}{\partial v_j(n)}
\end{equation}

\subsection{\textit{Listing}}

No \LaTeX\, \textit{listings} são um ambiente específico para exibição de código-fonte de forma formatada. Eles são gerenciados pelo pacote listings, que permite destacar a sintaxe de diversas linguagens de programação, ajustar a formatação do código e até mesmo adicionar numeração de linhas. O \textit{listings} é muito útil para exibir trechos de código de maneira organizada e fácil de ler. Na Listagem~\ref{lst:json}, apresentamos um exemplo no formato JSON.

    \begin{lstlisting}[language=json,caption={Leiaute do Campo de Auditoria do Banco de Dados},label={lst:json},firstnumber=1]
    {
    	"audit": {
    		"id": "column ID",
    		"database": "database name",
    		"table": "table name",
    		"user_app":"EXEHDA's user",
    		"dml": {
    			"action": "action type, ex. DELETE",
    			"timestamp": "date and time",
    			"user": "database user",
    			"ip": "connection Ip Address"
    		},
    		"row": {
    			"old_row": [{data before movement}],
    			"new_row": [{data after movement}]
    		}
    	}
    }
    \end{lstlisting}  

     Na Listagem~\ref{lst:trigger}, apresentamos um exemplo de código SQL, que ilustra a criação e utilização de um \textit{trigger} para automação de operações no banco de dados.

    \begin{lstlisting}[language=albandesSQL,caption={Sintaxe do \textit{Trigger} do Banco de Dados Mysql},label={lst:trigger},firstnumber=1]
    CREATE
        [DEFINER = user]
        TRIGGER [IF NOT EXISTS] trigger_name
        trigger_time trigger_event
        ON tbl_name FOR EACH ROW
        [trigger_order]
        /* trigger_body */
    \end{lstlisting}   

    \subsection{\textit{Algorithm}}

    No LaTeX, o \textit{algorithm} é um ambiente utilizado para escrever e formatar algoritmos de forma estruturada e legível. Esse ambiente permite a descrição clara de procedimentos e fluxos lógicos, facilitando a compreensão de algoritmos por leitores e pesquisadores.

    A utilização do ambiente \textit{algorithm} no LaTeX é especialmente útil em documentos acadêmicos da área de computação, matemática e engenharia, onde a clareza e a padronização da apresentação de algoritmos são fundamentais para a correta interpretação dos conceitos descritos. 

    No Algorithm~\ref{alg:apriori}, é exemplificada a descrição do Algoritmo Apriori, um dos métodos mais conhecidos e eficientes para a descoberta de padrões frequentes, sendo amplamente utilizado para a mineração de regras de associação em grandes conjuntos de dados~~\cite{Agrawal1994}.

    \begin{algorithm}
    \caption{Apriori Algorithm}\label{alg:apriori}
    \hspace*{\algorithmicindent} \textbf{Input} D: Input Dataset \\
    \hspace*{\algorithmicindent} \hspace{0.85cm} minSup: minimum support threshold \\    
    \hspace*{\algorithmicindent} \textbf{Output} All 2 to \textit{k}-frequent itemsets
    
    \begin{algorithmic}[1]

    \State \textit{L}\textsubscript{\textit{1}} = \{\textit{1}-\textit{frequent itemset}\}

    \For{(k=2; $\textit{L}\textsubscript{k-1} \not= \varphi$; \textit{k}++  )}
        \State \textit{C}\textsubscript{\textit{k}} = \textit{apriori\_gen}(\textit{L}\textsubscript{\textit{k}-\textit{1}})

        \ForEach {\textit{transaction  t in D}}
            \State \textit{C}\textsubscript{\textit{t}} = \textit{subset(\textit{C}\textsubscript{\textit{k}},t)}
            \ForEach{\textit{C} in \textit{C}\textsubscript{\textit{t}}}
                \State \textit{c.count}++
            \EndFor
        \EndFor
        \State $ \textit{L}\textsubscript{\textit{k}} = \{\textit{c}  \in \textit{C}\textsubscript{\textit{k}} \mid  \textit{\textit{c.count}} \geq \textit{minSup} \}$ 

    \EndFor        
        
    \Return  \textit{U}\textsubscript{\textit{k}}\textit{L}\textsubscript{\textit{k}}     
 
    \end{algorithmic}
    \end{algorithm}


    No Algorithm~\ref{alg:busca}, é apresentado um exemplo da descrição do algoritmo de busca binária, um método eficiente utilizado para localizar um elemento em uma lista ordenada. Esse algoritmo funciona ao dividir repetidamente o intervalo de pesquisa pela metade, comparando o valor buscado com o elemento central da lista, e ajustando a busca para a metade que contém o valor até encontrá-lo ou concluir que ele não está presente~\cite{Cormen2009}.
    
    \begin{algorithm}
    \caption{Busca Binária}
    \label{alg:busca}
    \begin{algorithmic}[1]
        \Procedure{BuscaBinária}{array, elemento}
            \State esquerda $\gets$ 0
            \State direita $\gets$ tamanho do array - 1
            \While{esquerda $\leq$ direita}
                \State meio $\gets$ (esquerda + direita) / 2
                \If{array[meio] = elemento}
                    \State \textbf{retorne} meio
                \ElsIf{array[meio] $<$ elemento}
                    \State esquerda $\gets$ meio + 1
                \Else
                    \State direita $\gets$ meio - 1
                \EndIf
            \EndWhile
            \State \textbf{retorne} -1
        \EndProcedure
    \end{algorithmic}
\end{algorithm}

\section{ Resultados e Discussão}

Os resultados devem ser apresentados de forma clara e objetiva. Esta seção inclui:

\begin{itemize}
    \item Apresentação dos dados: Utilize tabelas, gráficos e diagramas para visualizar os resultados.

    \item  Análise quantitativa e qualitativa: Interprete os dados obtidos, comparando-os com métodos anteriores.

    \item Validação estatística: Se aplicável, inclua testes estatísticos para confirmar a validade dos resultados.

    \item Discussão crítica: Explique possíveis limitações e o impacto dos resultados para a área de estudo.

\end{itemize}

\section{ Conclusão e Trabalhos Futuros}

A conclusão deve sintetizar as principais descobertas do estudo e sugerir próximos passos. Deve incluir:

\begin{itemize}
    \item Resumo das contribuições: Reafirme os principais avanços trazidos pelo trabalho.

    \item Implicâncias práticas e teóricas: Discuta como os resultados podem ser aplicados na indústria ou em futuras pesquisas.

    \item Sugestões para trabalhos futuros: Aponte direções para pesquisas futuras, como melhorias no método, novas aplicações ou extensões do estudo.

\end{itemize}

