
\documentclass[
	% -- opções da classe memoir --
	article,			% indica que é um artigo acadêmico
	12pt,				% tamanho da fonte
	oneside,			% para impressão apenas frente. Oposto a twoside
	a4paper,			% tamanho do papel. 
	% -- opções da classe abntex2 --
	%chapter=TITLE,		% títulos de capítulos em letras maiúsculas
	%section=TITLE,		% títulos de seções em letras maiúsculas
	%subsection=TITLE,	% títulos de subseções em letras maiúsculas
	%subsubsection=TITLE % títulos de subsubseções em letras maiúsculas
	% -- opções do pacote babel --
	english,			% idioma adicional para hifenização
	brazil				% o último idioma é o principal do documento
	]{abntex2}
%---------
% Packages
%---------
\usepackage{newtxtext,newtxmath}    % fonte Times New Roman
\usepackage[T1]{fontenc}            % codificação dos caracteres no PDF
\usepackage[utf8]{inputenc}         % codificação do arquivo
\usepackage{graphicx}               % inclusão de figuras
\usepackage[alf,num]{abntex2cite}   % Para autor-data (alfabética) e citações ABNT sistema numérico
\usepackage{indentfirst}            % indentar 1o. parágrafo
\usepackage{parskip}                % espaço entre parágrafos
\usepackage{microtype} 			    % para melhorias de justificação
\usepackage{color}
\usepackage[section]{placeins}      % comando \FloatBarrier
\usepackage{amsmath}
\usepackage{enumitem}
\usepackage{caption, subcaption}
%\usepackage{natbib}
%\usepackage[authoryear]{natbib}     % Ativa a citação no formato autor-ano.
%\usepackage[authoryear,square]{natbib} % Ativa a citação no formato autor-ano e adiciona colchetes.
\usepackage[alf]{abntex2cite}  % Configuração para autor-data (alfabética)
\usepackage{lipsum}  % Para gerar texto de exemplo
\usepackage{tocbibind} % Garante que "Anexos" apareça no sumário corretamente
%---

% Títulos de capítulos, seções etc. em fonte serifada
\renewcommand{\ABNTEXchapterfont}{\rmfamily}

% -------------------------
% Formatação de referências
% -------------------------
\citebrackets[]
\makeatletter
\renewcommand{\@biblabel}[1]{[#1]\quad}
\makeatother
\AtBeginDocument{\citeoption{abnt-emphasize=bf}}
\renewcommand{\citeonline}[1]{\citeauthoronline{#1} \cite{#1}}
% ---

% --------------------
% Configurações do PDF
% --------------------
\makeatletter
\hypersetup{
    pdftitle={\@title}, 
    pdfauthor={\@author},
    pdfsubject={},
    pdfcreator={LaTeX with abnTeX2},
    pdfkeywords={}, 
    colorlinks=true,    % false: boxed links; true: colored links
    linkcolor=black,    % color of internal links
    citecolor=black,    % color of links to bibliography
    filecolor=black,    % color of file links
    urlcolor=black,
    bookmarksdepth=4
}
\makeatother
% ---

% -------------------------
% Altera as margens padrões
% -------------------------
\setlrmarginsandblock{3cm}{3cm}{*}
\setulmarginsandblock{3cm}{3cm}{*}
\checkandfixthelayout
% ---

% --------------------------------------
% Espaçamentos entre linhas e parágrafos 
% --------------------------------------
% Recuo da primeira linha do parágrafo
\setlength{\parindent}{1.3cm}
% Espaçamento entre um parágrafo e outro:
\setlength{\parskip}{0.2cm}
\OnehalfSpacing                 % espaço entre linhas 1,5
% ---

% Configuração pacote enumitem (itemize, enumerate)
\setlist{parsep=0pt, leftmargin=1.3cm}

%--------------
% Code Listing
%--------------

% - https://pt.overleaf.com/learn/latex/Code_listing 
\usepackage{listings}
\usepackage{xcolor}
\definecolor{codegreen}{rgb}{0,0.6,0}
\definecolor{codegray}{rgb}{0.5,0.5,0.5}
\definecolor{codepurple}{rgb}{0.58,0,0.82}
\definecolor{backcolour}{rgb}{0.95,0.95,0.92}
\colorlet{punct}{red!60!black}
\definecolor{background}{HTML}{EEEEEE}
\definecolor{delim}{RGB}{20,105,176}
\colorlet{numb}{magenta!60!black}

% --- Json e SQL ---

\lstdefinelanguage{json}{
    basicstyle=\fontsize{8}{10}\ttfamily,
    numberstyle=\scriptsize,
    stepnumber=1,
    numbersep=8pt,
    showstringspaces=false,
    breaklines=true,
    frame=lines,
    backgroundcolor=\color{background},
    literate=
     *{0}{{{\color{numb}0}}}{1}
      {1}{{{\color{numb}1}}}{1}
      {2}{{{\color{numb}2}}}{1}
      {3}{{{\color{numb}3}}}{1}
      {4}{{{\color{numb}4}}}{1}
      {5}{{{\color{numb}5}}}{1}
      {6}{{{\color{numb}6}}}{1}
      {7}{{{\color{numb}7}}}{1}
      {8}{{{\color{numb}8}}}{1}
      {9}{{{\color{numb}9}}}{1}
      {:}{{{\color{punct}{:}}}}{1}
      {,}{{{\color{punct}{,}}}}{1}
      {\{}{{{\color{delim}{\{}}}}{1}
      {\}}{{{\color{delim}{\}}}}}{1}
      {[}{{{\color{delim}{[}}}}{1}
      {]}{{{\color{delim}{]}}}}{1},
}

\lstdefinelanguage{albandesSQL}{
    language=SQL,
    keywordstyle=\color{blue},
    %basicstyle=\normalfont\ttfamily,
    basicstyle=\fontsize{8}{10}\ttfamily,
    numbers=none,
    numberstyle=\scriptsize,
    stepnumber=1,
    numbersep=8pt,
    showstringspaces=false,
    breaklines=true,
    frame=lines,
    backgroundcolor=\color{background},
    literate=
     *{0}{{{\color{numb}0}}}{1}
      {1}{{{\color{numb}1}}}{1}
      {2}{{{\color{numb}2}}}{1}
      {3}{{{\color{numb}3}}}{1}
      {4}{{{\color{numb}4}}}{1}
      {5}{{{\color{numb}5}}}{1}
      {6}{{{\color{numb}6}}}{1}
      {7}{{{\color{numb}7}}}{1}
      {8}{{{\color{numb}8}}}{1}
      {9}{{{\color{numb}9}}}{1}
      {:}{{{\color{punct}{:}}}}{1}
      {,}{{{\color{punct}{,}}}}{1}
      {\{}{{{\color{delim}{\{}}}}{1}
      {\}}{{{\color{delim}{\}}}}}{1}
      {[}{{{\color{delim}{[}}}}{1}
      {]}{{{\color{delim}{]}}}}{1}
}

\lstdefinelanguage{albandes_json}{
    %basicstyle=\normalfont\ttfamily,
    basicstyle=\fontsize{8}{10}\ttfamily,
    %numbers=left,
    numberstyle=\scriptsize,
    stepnumber=1,
    numbersep=8pt,
    showstringspaces=false,
    breaklines=true,
    frame=lines,
    backgroundcolor=\color{background}
}

% --- [Fim} Json e SQL ---


% --- Algoritmo ---
\usepackage{algcompatible}
% OR \usepackage{algorithmic}
\usepackage{algorithm}
\usepackage{algpseudocode}

\makeatletter
% Reinsert missing \algbackskip
\def\algbackskip{\hskip-\ALG@thistlm}
\makeatother

\usepackage{fixltx2e}
\usepackage{amsmath}
% --- [FIM] Algoritmo ---